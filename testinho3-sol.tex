\documentclass[a4paper,12pt]{article}
\usepackage{amsfonts,amsmath,amstext,amssymb,amsthm,color}
\usepackage[brazil,portuguese]{babel}
\usepackage[utf8]{inputenc}
\usepackage[T1]{fontenc}
\usepackage{enumerate}
\usepackage{mathrsfs}
\usepackage{pgf}
\usepackage{tikz,tikz-cd,pgfplots}









\usepackage{graphicx,graphpap}
\usetikzlibrary{calc,intersections,through,backgrounds,positioning,decorations.pathreplacing,decorations.markings}
\usetikzlibrary{arrows}
\pgfplotsset{compat=1.15}





%------- DEFINIÇÕES DE COMANDOS UTILIZADOS -----------------------------------
\def\R{\mathbb R}
\def\C{\mathbb C}
\def\N{\mathbb N}
\def\eqmod{\!\!\!\mod}
\def\herm#1{\langle #1\rangle}
\def\hmod#1{\parallel #1\parallel}
\newcommand{\dps}{\displaystyle}
\newcommand{\bn}{\bigskip\noindent}
\newcommand{\mb}{\mathbb}
\newcommand{\mc}{\mathcal}
\newcommand{\mf}{\mathfrak}
\newcommand{\mtt}{\mathtt}
\def\ang{{\rm ang}}
\def\id{{\rm id}}
\def\sen{{\rm sen\ }}
\def\diag{{\rm diag}}
\def\dist{{\rm dist}}
\def\sdo{\raisebox{.06cm}{$\bigcirc$\hspace{-0.38cm}\raisebox{0.0cm}{$\bot$}}\,}
\def\spen{{\rm span}}
\def\rectanglepath{-- ++(1cm,0cm) -- ++(0cm,1cm) -- ++(-1cm,0cm) -- cycle}
\newcommand{\del}{\partial}
\newcounter{exn}
\setcounter{exn}{1}
\newcommand{\exn}{\theexn\stepcounter{exn}}
\newcommand{\rin}{\rotatebox[origin=c]{-90}{\Large $\in$}}
\newcommand{\rsubset}{\rotatebox[origin=c]{-90}{\Large $\subset$}}
\newcommand{\Lsubset}{\rotatebox[origin=c]{0}{\Large $\subset$}}
\newcommand{\pint}[2]{\langle #1,#2\rangle}
	
\renewcommand{\hom}{\mathrm{Hom}}

\newenvironment{sol}{\noindent\normalsize {\sc Solução:}}

\DeclareMathOperator{\Ker}{Ker}
\DeclareMathOperator{\im}{Im}






%------ Desenha ângulos retos no espaço ---------------------------------------------------------------------
%--- Parâmetros (A,B,C,t,s)
%--- sendo A, B, C pontos no epaço, e t e s números reais entre 0 e 1.
%--- t é a fração do segmento AB, e s é a fração do segmento BC utilizadas para contruir o quadradinho.
%------------------------------------------------------------------------------------------------------------
\newcommand\drawanguloreto[5]{
  \draw[-] ($#2 - #4*#2 + #4*#1$)  -- ($#2 - #4*#2 + #4*#1 - #5*#2 + #5*#3$) -- ($#2 - #5*#2 + #5*#3$);
}

\tikzset{
  % style to apply some styles to each segment of a path
  on each segment/.style={
    decorate,
    decoration={
      show path construction,
      moveto code={},
      lineto code={
        \path [#1]
        (\tikzinputsegmentfirst) -- (\tikzinputsegmentlast);
      },
      curveto code={
        \path [#1] (\tikzinputsegmentfirst)
        .. controls
        (\tikzinputsegmentsupporta) and (\tikzinputsegmentsupportb)
        ..
        (\tikzinputsegmentlast);
      },
      closepath code={
        \path [#1]
        (\tikzinputsegmentfirst) -- (\tikzinputsegmentlast);
      },
    },
  },
  % style to add an arrow in the middle of a path
  mid arrow/.style={postaction={decorate,decoration={
        markings,
        mark=at position .5 with {\arrow[#1]{stealth}}
      }}},
}
%------------------------------------------------------------------------------------------------------------




%--------  AJUSTANDO O TAMANHO DAS PÁGINAS -----------------------------------------------------
\addtolength{\textwidth}{4 cm}
\addtolength{\textheight}{3 cm}
\addtolength{\oddsidemargin}{-2 cm}
\addtolength{\evensidemargin}{-2 cm}
\addtolength{\topmargin}{-3 cm}



%-------- NUMERAÇÃO DE DEFINIÇÕES, TEOREMAS, ETC...  ---------------------------------------------
\newtheorem{df}{Definição}[subsection]
\newtheorem{thm}[df]{Teorema}
\newtheorem{cor}[df]{Corolário}
\newtheorem{prop}[df]{Proposição}
\newtheorem{lemma}[df]{Lema}
\newtheorem{conjec}[df]{Conjectura}
\newtheorem{exerc}[df]{Exercício(s)}
\newtheorem{qst}{Questão}
\newtheorem{ex}[df]{Exemplo(s)}
\newtheorem{enunc}{Enunciado}[exn]

%--------  TÍTULO E DATA   ----------------------------------------------------------
\author{``3º Testinho'' - Solucionário - GAAL}
\date{18 de Junho de 2019}
\title{}





%-------------------------------------------------------------------------------------------
%-------------------------------------------------------------------------------------------
%-----    INÍCIO DO TEXTO   ----------------------------------------------------------------
%-------------------------------------------------------------------------------------------
%-------------------------------------------------------------------------------------------
\begin{document}
\maketitle

Em todas as questões abaixo deixe sempre explícito o seu raciocínio, não apenas a resposta correta.

\begin{qst}
Para cada equação abaixo, diga que tipo de cônica ele representa e faça um esboço dessa cônica.
\begin{enumerate}[a)]
	\item $x=9y^2$;
	\item $4x^2+9x^2=36$;
	\item $x^2-\dfrac{y^2}{16}=-1$.
\end{enumerate}
\end{qst}
\begin{sol}
	\[\definecolor{ffqqqq}{rgb}{1.,0.,0.}
	\definecolor{qqffqq}{rgb}{0.,1.,0.}
	\definecolor{qqqqff}{rgb}{0.,0.,1.}
	\begin{tikzpicture}[line cap=round,line join=round,>=triangle 45]
	\begin{axis}[
	x=.5cm,y=.5cm,
	axis lines=middle,
	xmin=-4,
	xmax=8,
	ymin=-10,
	ymax=10,
	xtick={-2.0,0.0,...,8.0},
	ytick={-10.0,-8.0,...,10.0},]
	\clip(-4,-10) rectangle (8,10);
	\draw [samples=50,rotate around={-90.:(0.,0.)},xshift=0.cm,yshift=0.cm,color=qqqqff,domain=-5.0:5.0)] plot (\x,{(\x)^2/2/0.5});
	\draw [rotate around={0.:(0.,0.)},color=qqffqq] (0.,0.) ellipse (1.5cm and 1cm);
	\draw [samples=50,domain=-pi/3:pi/3,rotate around={90.:(0,0)},xshift=0.cm,yshift=0.cm,color=ffqqqq] plot ({4*(1+(\x)^2)/(1-(\x)^2)},{2*(\x)/(1-(\x)^2)});
	\draw [samples=50,domain=-pi/3:pi/3,rotate around={90.:(0,0)},xshift=0.cm,yshift=0.cm,color=ffqqqq] plot ({4*(-1-(\x)^2)/(1-(\x)^2)},{(-2)*(\x)/(1-(\x)^2)});
	\end{axis}
	\end{tikzpicture}\]onde \textcolor{red}{a equação (c) é uma hipérbole}, \textcolor{green}{a equação (b) é uma elipse} e \textcolor{blue}{a equação (a) é uma parábola}.
\end{sol}
\pagebreak
\begin{qst}
	Diga qual das equações abaixo descreve uma cônica rotacionada e qual descreve uma cônica transladada. Em seguida, apresente a matriz de rotação para a cônica rotacionada e o vetor de translação para a cônica transladada.
	\begin{enumerate}[a)]
		\item $9x^2-4xy+6y^2=30$;
		\item $4x^2-9y^2-8x-36y=76$.
	\end{enumerate}
\end{qst}
\begin{sol}
	A equação (a) descreve uma cônica rotacionada, devido à presença do termo $xy$. Ela pode ser reescrita como
	\[\begin{pmatrix}
	x&y
	\end{pmatrix}\begin{pmatrix}
	9&-2\\-2&6
	\end{pmatrix}\begin{pmatrix}
	x\\y
	\end{pmatrix}=30\tag{*}\] A matriz $\begin{pmatrix}
	9&-2\\-2&6
	\end{pmatrix}$ possui como autovalores as soluções da equação
	\[\det\begin{pmatrix}
	9-\lambda&-2\\-2&6-\lambda
	\end{pmatrix}=0\]que se traduz em $(9-\lambda)(6-\lambda)-4=0$, ou seja, $\lambda^2-15\lambda+50=0$. Ou seja, os autovalores da matriz são da forma
	\[\lambda=\frac{15\pm\sqrt{225-200}}{2}=\frac{15\pm 5}{2}\] ou seja, $\lambda=10$ e $\lambda=5$ são os nossos autovalores.
	
	Como os autovetores são as soluções não-triviais do sistema
	\[\begin{pmatrix}
	9-\lambda&-2\\-2&6-\lambda
	\end{pmatrix}\begin{pmatrix}
	x\\y
	\end{pmatrix}=0,\]podemos calcular os autovetores substituindo os autovalores obtidos acima, por exemplo $\lambda=5$:
	\[\begin{pmatrix}
	9-5&-2\\-2&6-5
	\end{pmatrix}\begin{pmatrix}
	x\\y
	\end{pmatrix}=0\]ou seja, queremos achar as soluções do sistema
	\[\begin{pmatrix}
	4&-2\\-2&1
	\end{pmatrix}\begin{pmatrix}
	x\\y
	\end{pmatrix}=0.\] Note, contudo, que as linhas da matriz são múltiplas uma da outra, e, portanto, qualquer uma delas nos dá todas as soluções: $W_1=\{-2x+y=0\}$, ou, visto de outra forma, $W_1=\{y=2x\}$. Escolhendo $v=(1,2)\in W_1$, vemos que $\lVert v\rVert = \sqrt{1^2+2^2}=\sqrt{5}$ e, portanto, podemos escolher o autovetor unitário $v_1=(1/\sqrt{5},2/\sqrt{5})$ associado ao autovalor $5$.
	
	Como os autoespaços de uma matriz simétrica são ortogonais, podemos tomar $v_2=(-2/\sqrt{5},1/\sqrt{5})$ e obter a matriz de rotação
	\[P=\begin{pmatrix}
	v_1&v_2
	\end{pmatrix}=\frac{1}{\sqrt5}\begin{pmatrix}
	1&-2\\
	2&1
	\end{pmatrix}.\]Por fim, reescrevendo a equação (*) usando a matriz de autovalores e a matriz de rotação obtemos
	\[\begin{pmatrix}
	x&y
	\end{pmatrix}P\begin{pmatrix}
	5&0\\0&10
	\end{pmatrix}P^t\begin{pmatrix}
	x\\y
	\end{pmatrix}=30\] ou seja,
	\[\begin{pmatrix}
	x'&y'
	\end{pmatrix}\begin{pmatrix}
	5&0\\0&10
	\end{pmatrix}\begin{pmatrix}
	x'\\y'
	\end{pmatrix}=30,\]donde $5x'^2+10y'^2=30$ (fazendo $(x',y')=P^t(x,y)$) e vemos que a cônica descrita por (a) está apenas rotacionada pela matriz de rotação $P$.
	
	\bigskip
	A equação (b), por sua vez, não possui o termo $xy$ e, portanto, não está rotacionada. Para descobrirmos a translação, temos que completar quadrados:
	\begin{align*}
		4x^2-9y^2-8x-36y=76\\
		4(x^2-2x)-9(y^2-4y)=76 & \quad\text{ agrupando os termos com $x$ e $y$}\\
		4((x-1)^2-1)-9((y-2)^2-4)=76 & \quad\text{ completando os quadrados}\\
		4(x-1)^2-4-9(y-2)^2+36=76 & \quad\text{ distribuindo $4$ e $9$}\\
		4x'^2-9y'^2=36 & \quad\text{ substituindo $(x',y')=(x-1,y-2)$}
	\end{align*}e vemos que o vetor translação é exatamente $T=(1,2)$.
\end{sol}\pagebreak

\begin{qst}	
	Dada a equação \[13 x^2 + 10 x y - 22 x + 13 y^2 + 58 y + 37 = 0\]faça o que se pede:
	\begin{enumerate}[a)]
		\item Reescreva equação acima em forma matricial.
		\item Encontre os autovalores e autovetores associados à parte de grau 2 da equação acima.
		\item Identifique a matriz de rotação associada a essa equação.
		\item Faça a mudança de coordenadas que diagonaliza a forma matricial obtida no item (a).
		\item Identifique a translação associada a essa equação.
		\item Faça uma translação de forma que a equação obtida no item (d) esteja na forma padrão.
		\item Identifique qual a cônica descrita pela equação.
		\item Faça um esboço dessa cônica.
	\end{enumerate}
\end{qst}
\begin{sol}
	Começamos re-escrevendo a equação na forma
	\[\begin{pmatrix}
	x&y
	\end{pmatrix}\begin{pmatrix}
	13 & 5\\5&13
	\end{pmatrix}\begin{pmatrix}
	x\\y
	\end{pmatrix}+\begin{pmatrix}
	-22&58
	\end{pmatrix}\begin{pmatrix}
	x\\y
	\end{pmatrix}+37=0\tag{$\ast$}.\]Vamos agora diagonalizar a matriz $A=\begin{pmatrix}
	13&5\\5&13
	\end{pmatrix}$. Para isso, vamos calcular os autovalores como sendo as soluções da equação
	\[\det \begin{pmatrix}
	13-\lambda&5\\5&13-\lambda
	\end{pmatrix}=0,\]ou seja, $(13-\lambda)^2-25=0$, que podemos ainda re-escrever como
	\[\lambda^2-26\lambda+169-25=\lambda^2-26\lambda+144=0.\] Sabemos, ainda, que as soluções dessa equação são da forma
	\[\lambda=\frac{26\pm\sqrt{26^2-4\cdot144}}{2}=\frac{26\pm\sqrt{676-576}}{2}=\frac{26\pm10}{2},\] ou seja, $\lambda_1=18$ e $\lambda_2=8$.
	
	Com isso, podemos calcular os autoespaços $W_1$ e $W_2$ como sendo as soluções não-triviais da equação $(A-\lambda I)X=0$ para cada valor de $\lambda$. Para $\lambda_1$ temos:
	\[\begin{pmatrix}
	13-18 & 5\\5&13-18
	\end{pmatrix}\begin{pmatrix}
	x\\y
	\end{pmatrix}=0,\]ou seja,
	\[\begin{pmatrix}
	-5 & 5\\5&-5
	\end{pmatrix}\begin{pmatrix}
	x\\y
	\end{pmatrix}=0\]que admite solução da forma $W_1=\{-5x+5y=0\}$ ou seja, $W_1=\{x=y\}$. Assim, podemos pegar o vetor $v_1=(1/\sqrt 2,1/\sqrt2)\in W_1$ e ver que $\lVert v_1\rVert = \sqrt{(1/\sqrt2)^2+(1/\sqrt2)^2}=\sqrt{1/2+1/2}=\sqrt{1}=1$, ou seja, $v_1$ é um autovetor normal da matriz $A$ associado ao autovalor $18$.
	
	Como $A$ é simétrica, sabemos que $W_2=W_1^\perp$ e, portanto, podemos simplesmente tomar $v_2=(-1/\sqrt 2,1/\sqrt 2)$.
	
	Claramente $\lVert v_2\rVert = 1$ (já que $\lVert v_2\rVert = \lVert v_1\rVert$) e $v_1\perp v_2$ (por construção). Segue que $A=PDP^t$ onde $P=\begin{pmatrix}
	v_1 & v_2
	\end{pmatrix}=\frac{1}{\sqrt2}\begin{pmatrix}
	1&-1\\1&1
	\end{pmatrix}$ e $D=\begin{pmatrix}
	18&0\\0&8
	\end{pmatrix}$.
	
	Com isso, vamos reescrever ($\ast$):
	
	\[\begin{pmatrix}
	x&y
	\end{pmatrix}PDP^t\begin{pmatrix}
	x\\y
	\end{pmatrix}+\begin{pmatrix}
	-22&58
	\end{pmatrix}\begin{pmatrix}
	x\\y
	\end{pmatrix}+37=0\]e, fazendo a mudança de coordenadas $\begin{pmatrix}
	x&y
	\end{pmatrix}P=\begin{pmatrix}
	x' & y'
	\end{pmatrix}$ temos que $\begin{pmatrix}
	x'\\y'
	\end{pmatrix}=P^t\begin{pmatrix}
	x\\y
	\end{pmatrix}$ e, portanto, multiplicando à esquerda por $P$ ambos os lados da equação acima, obtemos
	$P\begin{pmatrix}
	x'\\y'
	\end{pmatrix}=\begin{pmatrix}
	x\\y
	\end{pmatrix}$. Assim, obtemos
	\[\begin{pmatrix}
	x'&y'
	\end{pmatrix}D\begin{pmatrix}
	x'\\y'
	\end{pmatrix}+\begin{pmatrix}
	-22 & 58
	\end{pmatrix}P\begin{pmatrix}
	x'\\y'
	\end{pmatrix}+37=0.\] Antes de avançar, vamos calcular $\begin{pmatrix}
	-22&58
	\end{pmatrix}P$:
	\[\begin{pmatrix}
	-22&58
	\end{pmatrix}P=\begin{pmatrix}
	-22&58
	\end{pmatrix}\frac{1}{\sqrt2}\begin{pmatrix}
	1&-1\\1&1
	\end{pmatrix}=\frac{1}{\sqrt2}\begin{pmatrix}
	36&80
	\end{pmatrix}\]e escrever, finalmente,
	\[\begin{pmatrix}
	x'&y'
	\end{pmatrix}\begin{pmatrix}
	18&0\\0&8
	\end{pmatrix}\begin{pmatrix}
	x'\\y'
	\end{pmatrix}+1/\sqrt{2}\begin{pmatrix}
	36&80
	\end{pmatrix}\begin{pmatrix}
	x'\\y'
	\end{pmatrix}+37=0\tag{$\ast\ast$}.\]
	
	Agora, obtemos a equação $18x'^2+8y'^2+36/\sqrt{2}x'+80/\sqrt{2}y'+37=0$ que podemos completar quadrados:
	\begin{align*}
		18x'^2+8y'^2+36/\sqrt{2}x'+80/\sqrt{2}y'+37=0\\
		(18x'^2+36/\sqrt{2}x')+(8y'^2+80/\sqrt{2}y')+37=0&\quad\text{ agrupando $x'$ e $y'$}\\
		18(x'^2+2/\sqrt{2}x)+8(y'^2+10/\sqrt{2}y')+37=0&\quad\text{ colocando $18$ e $8$ em evidência}\\
		18((x'+1/\sqrt{2})^2-1/2)+8((y'+5/\sqrt{2})^2-25/2)&\quad\text{ completando os quadrados}\\
		18(x'+1/\sqrt{2})^2-9+8(y'+5/\sqrt{2})^2-100+37=0&\quad\text{ distribuindo $18$ e $8$}\\
		18x''^2-9+8y''^2-100+37=0&\quad\text{ fazendo $\begin{pmatrix}
			x''\\y''
			\end{pmatrix}=\begin{pmatrix}
			x'+1/\sqrt{2}\\y'+5/\sqrt{2}
			\end{pmatrix}$}\\
		18x''^2+8y''^2=72&\quad\text{ fazendo as operações com números restantes}
	\end{align*}o que nos diz que fazendo a mudança de coordenadas $\begin{pmatrix}
	x''\\y''
	\end{pmatrix}=\begin{pmatrix}
	x'+1/\sqrt{2}\\y'+5/\sqrt{2}
	\end{pmatrix}$ podemos re-escrever ($\ast\ast$) como
	\[18x''^2+8y''^2=72\tag{$\ast\ast\ast$}.\]
	
	Finalmente, dividindo ambos os lados por $72$ obtemos
	\[\frac{x''^2}{4}+\frac{y''^2}{9}=1\]e, portanto, concluímos que a equação ($\ast$) descreve uma elipse de semi-eixos $2$ e $3$. 
	
	Para esboçar a cônica, precisamos achar o quanto ela está rodada e transladada. Achar a rotação é fácil: Basta notar que $P=R_{45^\circ}$.
	
	 Para descobrir a translação, note que já sabemos o quanto o eixo $(x'',y'')$ está transladado em relação ao eixo $(x',y'')$: exatamente $(-1/\sqrt{2},-5/\sqrt{2})$. Mas sabemos que $(x',y')=P^t(x,y)$, ou seja, a translação pode ser calculada como $P(-1/\sqrt2,-5/\sqrt{2})$:
	 \[\frac{1}{\sqrt{2}}\begin{pmatrix}
	 1&-1\\1&1
	 \end{pmatrix}\frac{1}{\sqrt{2}}\begin{pmatrix}
	 -1\\-5
	 \end{pmatrix}=\frac{1}{2}\begin{pmatrix}
	 4\\-6
	 \end{pmatrix}=\begin{pmatrix}
	 2\\-3
	 \end{pmatrix}\]o que nos dá uma translação de $T=(2,-3)$, e podemos esboçar nossa cônica:
	 \[\definecolor{qqwuqq}{rgb}{0.,0.39215686274509803,0.}
	 \definecolor{uuuuuu}{rgb}{0.26666666666666666,0.26666666666666666,0.26666666666666666}
	 \begin{tikzpicture}[line cap=round,line join=round,>=triangle 45,x=1.0cm,y=1.0cm]
	 \begin{axis}[
	 x=1.0cm,y=1.0cm,
	 axis lines=middle,
	 xmin=-1.9220943849833594,
	 xmax=6.007070171729744,
	 ymin=-6.020931362186028,
	 ymax=2.0498254187540863,
	 xtick={-1.0,0.0,...,6.0},
	 ytick={-6.0,-5.0,...,2.0},]
	 \clip(-1.9220943849833594,-6.020931362186028) rectangle (6.007070171729744,2.0498254187540863);
	 \draw [shift={(5.,0.)},color=qqwuqq,fill=qqwuqq,fill opacity=0.10000000149011612] (0,0) -- (0.:0.47197408075673236) arc (0.:45.:0.47197408075673236) -- cycle;
	 \draw [rotate around={-45.:(2.,-3.)},] (2.,-3.) ellipse (3.cm and 2.cm);
	 \draw [->,] (0.,0.) -- (2.,-3.);
	 \draw [line width=1.2pt,dotted,domain=-1.9220943849833594:6.007070171729744] plot(\x,{(-1.-1.*\x)/1.});
	 \draw [line width=1.2pt,dotted,domain=-1.9220943849833594:6.007070171729744] plot(\x,{(-5.--1.*\x)/1.});
	 \draw [->,] (2.,-3.) -- (2.7071067811865475,-2.2928932188134525);
	 \draw [->,] (2.,-3.) -- (1.2928932188134525,-2.2928932188134525);
	 \begin{scriptsize}
	 \draw [fill=uuuuuu] (2.,-3.) circle (2.0pt);
	 \draw [fill=uuuuuu] (0.,0.) circle (2.0pt);
	 \draw[color=black] (1.5,-1.3247892586565462) node {$T$};
	 \draw[color=black] (2.522328208809204,-3) node {$v_1$};
	 \draw[color=black] (1.5626475779371811,-3) node {$v_2$};
	 \draw[color=qqwuqq] (5.5,0.1697953304064379) node {$45^\circ$};
	 \end{scriptsize}
	 \end{axis}
	 \end{tikzpicture}\]
\end{sol}

\end{document}