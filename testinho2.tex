\documentclass[a4paper,12pt]{article}
\usepackage{amsfonts,amsmath,amstext,amssymb,amsthm,color}
\usepackage[brazil,portuguese]{babel}
\usepackage[utf8]{inputenc}
\usepackage[T1]{fontenc}
\usepackage{enumerate}
\usepackage{pgf}
\usepackage{tikz,tikz-cd}









\usepackage{graphicx,graphpap}
\usetikzlibrary{calc,intersections,through,backgrounds,positioning,decorations.pathreplacing,decorations.markings}





%------- DEFINIÇÕES DE COMANDOS UTILIZADOS -----------------------------------
\def\R{\mathbb R}
\def\C{\mathbb C}
\def\N{\mathbb N}
\def\eqmod{\!\!\!\mod}
\def\herm#1{\langle #1\rangle}
\def\hmod#1{\parallel #1\parallel}
\newcommand{\dps}{\displaystyle}
\newcommand{\bn}{\bigskip\noindent}
\newcommand{\mb}{\mathbb}
\newcommand{\mc}{\mathcal}
\newcommand{\mf}{\mathfrak}
\newcommand{\mtt}{\mathtt}
\def\ang{{\rm ang}}
\def\id{{\rm id}}
\def\sen{{\rm sen\ }}
\def\diag{{\rm diag}}
\def\dist{{\rm dist}}
\def\sdo{\raisebox{.06cm}{$\bigcirc$\hspace{-0.38cm}\raisebox{0.0cm}{$\bot$}}\,}
\def\spen{{\rm span}}
\def\rectanglepath{-- ++(1cm,0cm) -- ++(0cm,1cm) -- ++(-1cm,0cm) -- cycle}
\newcommand{\del}{\partial}
\newcounter{exn}
\setcounter{exn}{1}
\newcommand{\exn}{\theexn\stepcounter{exn}}
\newcommand{\rin}{\rotatebox[origin=c]{-90}{\Large $\in$}}
\newcommand{\rsubset}{\rotatebox[origin=c]{-90}{\Large $\subset$}}
\newcommand{\Lsubset}{\rotatebox[origin=c]{0}{\Large $\subset$}}
\newcommand{\pint}[2]{\langle #1,#2\rangle}
	
\renewcommand{\hom}{\mathrm{Hom}}

\newenvironment{sol}{\noindent\normalsize {\sc Solução:}}

\DeclareMathOperator{\Ker}{Ker}
\DeclareMathOperator{\im}{Im}






%------ Desenha ângulos retos no espaço ---------------------------------------------------------------------
%--- Parâmetros (A,B,C,t,s)
%--- sendo A, B, C pontos no epaço, e t e s números reais entre 0 e 1.
%--- t é a fração do segmento AB, e s é a fração do segmento BC utilizadas para contruir o quadradinho.
%------------------------------------------------------------------------------------------------------------
\newcommand\drawanguloreto[5]{
  \draw[-] ($#2 - #4*#2 + #4*#1$)  -- ($#2 - #4*#2 + #4*#1 - #5*#2 + #5*#3$) -- ($#2 - #5*#2 + #5*#3$);
}

\tikzset{
  % style to apply some styles to each segment of a path
  on each segment/.style={
    decorate,
    decoration={
      show path construction,
      moveto code={},
      lineto code={
        \path [#1]
        (\tikzinputsegmentfirst) -- (\tikzinputsegmentlast);
      },
      curveto code={
        \path [#1] (\tikzinputsegmentfirst)
        .. controls
        (\tikzinputsegmentsupporta) and (\tikzinputsegmentsupportb)
        ..
        (\tikzinputsegmentlast);
      },
      closepath code={
        \path [#1]
        (\tikzinputsegmentfirst) -- (\tikzinputsegmentlast);
      },
    },
  },
  % style to add an arrow in the middle of a path
  mid arrow/.style={postaction={decorate,decoration={
        markings,
        mark=at position .5 with {\arrow[#1]{stealth}}
      }}},
}
%------------------------------------------------------------------------------------------------------------




%--------  AJUSTANDO O TAMANHO DAS PÁGINAS -----------------------------------------------------
\addtolength{\textwidth}{4 cm}
\addtolength{\textheight}{3 cm}
\addtolength{\oddsidemargin}{-2 cm}
\addtolength{\evensidemargin}{-2 cm}
\addtolength{\topmargin}{-3 cm}



%-------- NUMERAÇÃO DE DEFINIÇÕES, TEOREMAS, ETC...  ---------------------------------------------
\newtheorem{df}{Definição}[subsection]
\newtheorem{thm}[df]{Teorema}
\newtheorem{cor}[df]{Corolário}
\newtheorem{prop}[df]{Proposição}
\newtheorem{lemma}[df]{Lema}
\newtheorem{conjec}[df]{Conjectura}
\newtheorem{exerc}[df]{Exercício(s)}
\newtheorem{qst}{Questão}
\newtheorem{ex}[df]{Exemplo(s)}
\newtheorem{enunc}{Enunciado}[exn]

%--------  TÍTULO E DATA   ----------------------------------------------------------
\author{``2º Testinho'' - GAAL}
\date{02 de Maio de 2019}
\title{}





%-------------------------------------------------------------------------------------------
%-------------------------------------------------------------------------------------------
%-----    INÍCIO DO TEXTO   ----------------------------------------------------------------
%-------------------------------------------------------------------------------------------
%-------------------------------------------------------------------------------------------
\begin{document}
\maketitle

Em todas as questões abaixo, sempre que encontrar uma solução você deve mostrar que ela é, de fato, uma solução.

\begin{qst}
Considere a função $f:\R^2\to \R^2$ definida por $f(x,y)=\frac{1}{2}(x+y,x+y)$.
\begin{enumerate}[a)]
	\item Mostre que $f$ é linear.
	\item Calcule $\Ker f$ e $\im f$.
	\item Exiba um conjunto de geradores l.i. para $\Ker f$ e $\im f$.
	\item Calcule o produto interno de qualquer gerador de $\Ker f$ com qualquer gerador de $\im f$ escolhidos acima.
	\item Conclua que $\Ker f\perp\im f$, isto é, qualquer elemento de $\Ker f$ é ortogonal a qualquer elemento de $\im f$.
\end{enumerate}
\end{qst}

\begin{qst}
Considere o conjunto de vetores $\{(1,1), (2,3), (3,5), (-1, 5)\}$ em $\R^2$.
\begin{enumerate}[a)]
	\item Mostre que esse conjunto é l.d.
	\item Mostre que os vetores $(1,0)$ e $(0,1)$ são gerado por esse conjunto.
	\item Calcule o subespaço de $\R^2$ gerado por esse conjunto.
	\item É possível encontrar uma subcoleção desses vetores com dois elementos e que seja l.i.? Se sim, exiba tal subcoleção, se não, prove que é impossível.
\end{enumerate}
\end{qst}

\begin{qst}
	Considere os vetores $v=(1,1,1)$, $u=(1,0,-1)$ e $w=(-1,-2,3)$ em $\R^3$.
	\begin{enumerate}[a)]
		\item Calcule os produtos internos $\pint{v}{u}$, $\pint v w$ e $\pint u w$.
		\item Usando o item acima, conclua que existe um plano $\pi$ que é gerado por dois dos vetores acima e é ortogonal ao terceiro. Explicite quais são os vetores que geram o plano e qual é o vetor ortogonal ao plano.
		\item Calcule o produto vetorial dos vetores que geram o plano e compare o resultado com o vetor ortogonal ao plano.
	\end{enumerate}
\end{qst}
\begin{qst}
Considere os planos $$\pi_1=\{v\in \R^3\mid v=\lambda(1,1,1)+\mu(1,0,-1)+(0,0,3),\mbox{ com }\lambda\mbox{ e }\mu\in \R\}$$ $$\pi_2=\{(x,y,z)\in\R^3\mid x-2y+7z=5\}.$$
\begin{enumerate}[a)]
	\item Calcule a interseção desses planos.
	\item Explicite um conjunto de geradores l.i. dessa interseção.
\end{enumerate}
\end{qst}

\end{document}