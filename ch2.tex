\chapter{Real Linear Algebra}
\section{Introduction}
\subsection{First notions and definitions}

To start working with vector spaces we first need to understand that the perspective is going to change a bit from the previous chapter. We're leaving the domain of \textbf{set theory} and jumping right in the domain of \textbf{algebra}.

\textit{Algebra }is the domain of mathematics that deals with operations and their properties.

\begin{df}
	Given any set $X$, a \textbf{(binary) operation} on $X$ is a function $f:X\times X \to X$.
\end{df}

\begin{ex}
	Let $\N$ be the set of natural numbers, as before. We have a few operations here:
	\[f,g,h:\N\times\N \to \N\]
	\[(n,m)\mapsto f(n,m):=n+m\]
	\[(n,m)\mapsto g(n,m):=nm\]
	\[(n,m)\mapsto h(n,m):=n^m\]and some of these operations have some properties that the others don't.
	
	For instance, all three functions satisfy the following property:
	
	\begin{itemize}
		\item Let $\phi$ be an operation on $X$. There is some $n_e\in X$ such that $\phi(n,n_e)=n$ for all $n\in X$.
	\end{itemize}

	In the case of $f$, if we choose $n_e:=0$, we see that $f(n,0)=n+0=n$, no matter which $n\in \N$ we chose, so $f$ satisfies the property above.
	
	In the case of $g$, if we choose $n_e:=1$, we see that $g(n,1)=n\cdot1=n$, no matter which $n\in\N$ we chose, so $g$ satisfies the property above.
	
	Finally, in the case of $h$, if we choose $n_e:=1$, we see that $h(n,1)=n^1=n$, no matter which $n\in\N$ we chose, so $h$ satisfies the property above.
	
	\bigskip
	Next up is the property:
	
	\begin{itemize}
		\item Let $\phi$ be an operation on $X$. There is some $n_e\in X$ such that $\phi(n_e,n)=n$ for all $n\in X$.
	\end{itemize}

	What can we say about $f,g,h$ in this case? Well, it's easy to see that for $f$ and $g$ it still holds true - and it does so for the same value of $n_e$ as before.
	
	However, for $h$ it fails. For instance, is there some number $x\in\N$ such that $h(x,2)=2$? Well, by definition of $h$ we would need to have $x^2=2$ and so $x=\sqrt{2}$ which is not in $\N$ - this tells us that there's no such $x\in\N$. It follows that this property fails for $h$.
	
	\bigskip
	Summing up all of these together, we get the following property:
	\begin{itemize}
		\item (Identity element) Let $\phi$ be an operation on $X$. There is some $n_e\in X$ such that $\phi(n,n_e)=n=\phi(n_e,n)$ for all $n\in X$.
	\end{itemize}
	And we see that $f$ and $g$ have what's called an \textit{identity element} - it's an element $n_e$ such that if you fix it in any input of your operation, then your operation is just the identity function.
	
	\bigskip
	Consider now the following property:
	\begin{itemize}
		\item (Associativity) Let $\phi$ be an operation on $X$. Then, for all $n,m,l\in X$ we have that $\phi(\phi(n,m),l)=\phi(n,\phi(m,l))$.
	\end{itemize}

	In the case of $f$ we can check
	\[f(f(n,m),l)=f(n+m,l)=(n+m)+l=n+(m+l)=f(n,m+l)=f(n,f(m,l))\]and see that $f$ is associative.
	
	In the case of $g$ we can check
	\[g(g(n,m),l)=g(nm,l)=(nm)l=n(ml)=g(n,ml)=g(n,g(m,l))\]and see that $g$ is associative.
	
	However, for $h$, once again, this property fails: For instance, let us compare $h(h(2,2),3)$ and $h(2,h(2,3))$:
	\[h(h(2,2),3)=h(2^2,3)=(2^2)^3=4^3=64\]
	\[h(2,h(2,3))=h(2,2^3)=2^{(2^3)}=2^8=256\]so they are clearly different, and $h$ is not associative.
	
	\bigskip
	One more:
	\begin{itemize}
		\item (Commutativity) Let $\phi$ be an operation on $X$. Then, for all $n,m\in X$ we have that $\phi(n,m)=\phi(m,n)$.
	\end{itemize}

	In the case of $f$ we can easily see that $f(n,m)=n+m=m+n=f(m,n)$.
	
	Similarly for $g$, we see that $g(n,m)=nm=mn=g(m,n)$.
	
	But, once again, $h(2,3)=8\neq 9=h(3,2)$, so $h$ is not commutative.
	
	\bigskip
	These are the most common operations in $\N$ and some of their properties. Now, let us show something that is \textbf{not} an operation:
	
	Consider the functions $$f',g':\N\times\N\to\N$$\[(n,m)\mapsto f'(n,m):=n-m\]
	\[(n,m)\mapsto g'(n,m):=n/m.\]
	
	Notice that I've just lied to you - these are \textbf{not} functions. To see that, take $f'$ and apply it on $(3,1)$. By definition of function, $f'(3,1)$ should lie on $\N$, the codomain of $f'$. But, by definition of $f'$, we see that $f'(3,1)=3-1=-2$, which is \textbf{not} in $\N$.
	
	Similarly, $g'$ isn't a function for the same reason: It should take, for instance, $(1,2)$ to a natural number - but it doesn't. It takes $(1,2)$ to $g'(1,2)=1/2$ which, once more, is not a natural number.
	
	However, for \textit{some} specific values of the input, $f'$ and $g'$ really have outputs in $\N$. For that reason, they are called \textbf{partial operations} and, sadly, won't be studied in this text, since we're mostly concerned with proper operations.
	
	If, however, you'd like to learn more about partial operations, you should click \href{https://ncatlab.org/nlab/show/groupoid}{here} or Google for ``groupoid'' - which is precisely the mathematical notion of a set with an associative partial operation.
\end{ex}