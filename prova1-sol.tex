\documentclass[a4paper,10pt]{article}
\usepackage{amsfonts,amsmath,amstext,amssymb,amsthm,color}
\usepackage{thmtools}
\usepackage[brazil,portuguese]{babel}
\usepackage[utf8]{inputenc}
\usepackage[T1]{fontenc}
\usepackage{enumerate}
\usepackage{pgf}
\usepackage{tikz,tikz-cd}
\usepackage{arydshln}
\usepackage{tcolorbox}
\usepackage[bottom=3cm]{geometry}








\tcbuselibrary{breakable}
\usepackage{graphicx,graphpap}
\usetikzlibrary{calc,intersections,through,backgrounds,positioning,decorations.pathreplacing,decorations.markings}





%------- DEFINIÇÕES DE COMANDOS UTILIZADOS -----------------------------------
\def\R{\mathbb R}
\def\C{\mathbb C}
\def\N{\mathbb N}
\def\eqmod{\!\!\!\mod}
\def\herm#1{\langle #1\rangle}
\def\hmod#1{\parallel #1\parallel}
\newcommand{\dps}{\displaystyle}
\newcommand{\bn}{\bigskip\noindent}
\newcommand{\mb}{\mathbb}
\newcommand{\mc}{\mathcal}
\newcommand{\mf}{\mathfrak}
\newcommand{\mtt}{\mathtt}
\def\ang{{\rm ang}}
\def\id{{\rm id}}
\def\sen{{\rm sen\ }}
\def\diag{{\rm diag}}
\def\dist{{\rm dist}}
\def\sdo{\raisebox{.06cm}{$\bigcirc$\hspace{-0.38cm}\raisebox{0.0cm}{$\bot$}}\,}
\def\spen{{\rm span}}
\def\rectanglepath{-- ++(1cm,0cm) -- ++(0cm,1cm) -- ++(-1cm,0cm) -- cycle}
\newcommand{\del}{\partial}
\newcounter{exn}
\setcounter{exn}{1}
\newcommand{\exn}{\theexn\stepcounter{exn}}
\newcommand{\rin}{\rotatebox[origin=c]{-90}{\Large $\in$}}
\newcommand{\rsubset}{\rotatebox[origin=c]{-90}{\Large $\subset$}}
\newcommand{\Lsubset}{\rotatebox[origin=c]{0}{\Large $\subset$}}
\renewcommand{\hom}{\mathrm{Hom}}

\newenvironment{sol}{\begin{tcolorbox}[breakable,colback=blue!5!white,colframe=blue!40!white,title=\normalsize {\sc{Solução}},coltitle=black]}{\end{tcolorbox}}

\newenvironment{augmatrix}{\left(\begin{array}}{\end{array}\right)}






%------ Desenha ângulos retos no espaço ---------------------------------------------------------------------
%--- Parâmetros (A,B,C,t,s)
%--- sendo A, B, C pontos no epaço, e t e s números reais entre 0 e 1.
%--- t é a fração do segmento AB, e s é a fração do segmento BC utilizadas para contruir o quadradinho.
%------------------------------------------------------------------------------------------------------------
\newcommand\drawanguloreto[5]{
  \draw[-] ($#2 - #4*#2 + #4*#1$)  -- ($#2 - #4*#2 + #4*#1 - #5*#2 + #5*#3$) -- ($#2 - #5*#2 + #5*#3$);
}

\tikzset{
  % style to apply some styles to each segment of a path
  on each segment/.style={
    decorate,
    decoration={
      show path construction,
      moveto code={},
      lineto code={
        \path [#1]
        (\tikzinputsegmentfirst) -- (\tikzinputsegmentlast);
      },
      curveto code={
        \path [#1] (\tikzinputsegmentfirst)
        .. controls
        (\tikzinputsegmentsupporta) and (\tikzinputsegmentsupportb)
        ..
        (\tikzinputsegmentlast);
      },
      closepath code={
        \path [#1]
        (\tikzinputsegmentfirst) -- (\tikzinputsegmentlast);
      },
    },
  },
  % style to add an arrow in the middle of a path
  mid arrow/.style={postaction={decorate,decoration={
        markings,
        mark=at position .5 with {\arrow[#1]{stealth}}
      }}},
}
%------------------------------------------------------------------------------------------------------------




%--------  AJUSTANDO O TAMANHO DAS PÁGINAS -----------------------------------------------------
\addtolength{\textwidth}{4 cm}
\addtolength{\textheight}{3 cm}
\addtolength{\oddsidemargin}{-2 cm}
\addtolength{\evensidemargin}{-2 cm}
\addtolength{\topmargin}{-3 cm}



%-------- NUMERAÇÃO DE DEFINIÇÕES, TEOREMAS, ETC...  ---------------------------------------------
\newtheorem{qst}{Questão}
\declaretheorem[numbered=no, name=Questão Bônus]{bonus}
\newtheorem{enunc}{Enunciado}[exn]

%--------  TÍTULO E DATA   ----------------------------------------------------------
\author{Primeira Prova - GAAL}
\date{04 de Abril de 2019}
\title{}





%-------------------------------------------------------------------------------------------
%-------------------------------------------------------------------------------------------
%-----    INÍCIO DO TEXTO   ----------------------------------------------------------------
%-------------------------------------------------------------------------------------------
%-------------------------------------------------------------------------------------------
\begin{document}
\begin{center}
	{\Large{\sc Solucionário - Primeira Prova - GAAL}}
\end{center}

\begin{qst}
	Considere o sistema linear
	\[\begin{pmatrix}
	3&0&-1&0\\4&1&-1&-2\\0&1&0&-1
	\end{pmatrix}\begin{pmatrix}
	x\\y\\z\\w
	\end{pmatrix}=\begin{pmatrix}
	0\\0\\0
	\end{pmatrix}\]e faça o que se pede:
	\begin{enumerate}[a)]
		\item Exiba a matriz aumentada desse sistema.
		\item Exiba a matriz escalonada desse sistema.
		\item Calcule o conjunto solução desse sistema.
		\item Exiba uma solução do sistema, e verifique que ela é, de fato, uma solução.
	\end{enumerate}
\end{qst}
\begin{sol}
	\begin{enumerate}[a)]
		\item A matriz aumentada do sistema é
		\[\begin{augmatrix}{cccc:c}
		3&0&-1&0&0\\
		4&1&-1&-2&0\\
		0&1&0&-1&0
		\end{augmatrix}.\]
		\item Vamos escalonar:
		\begin{align*}
		\begin{augmatrix}{cccc:c}
		3&0&-1&0&0\\
		4&1&-1&-2&0\\
		0&1&0&-1&0
		\end{augmatrix}\rightsquigarrow\begin{augmatrix}{cccc:c}
		1&0&-1/3&0&0\\
		4&1&-1&-2&0\\
		0&1&0&-1&0
		\end{augmatrix}\rightsquigarrow\begin{augmatrix}{cccc:c}
		1&0&-1/3&0&0\\
		0&1&1/3&-2&0\\
		0&1&0&-1&0
		\end{augmatrix}\\\\
		\rightsquigarrow\begin{augmatrix}{cccc:c}
		1&0&-1/3&0&0\\
		0&1&1/3&-2&0\\
		0&0&-1/3&1&0
		\end{augmatrix}
		\rightsquigarrow\begin{augmatrix}{cccc:c}
		1&0&-1/3&0&0\\
		0&1&1/3&-2&0\\
		0&0&1&-3&0
		\end{augmatrix}\rightsquigarrow\begin{augmatrix}{cccc:c}
		1&0&0&-1&0\\
		0&1&0&-1&0\\
		0&0&1&-3&0
		\end{augmatrix}
		\end{align*}e obter que a matriz escalonada do sistema é \[\begin{augmatrix}{cccc:c}
		1&0&0&-1&0\\
		0&1&0&- 1&0\\
		0&0&1&-3&0
		\end{augmatrix}.\]
		\item Com a matriz escalonada, vemos que temos as equações $x=w$, $y=w$ e $z=3w$. Como $x,y,z$ dependem de $w$, vamos declarar $w$ variável livre e expressar o conjunto solução como
		\[S=\left\{(x,y,z,w)\in\R^4\mid x=y=z=3w\right\},\] (ou
		\[S=\left\{\lambda \begin{pmatrix}
		1\\1\\3\\1
		\end{pmatrix}\mid \lambda\in \R \right\}\]se você preferir).
		\item Para achar uma solução não-trivial, basta escolher qualquer $w\neq 0$, por exemplo, $w=1$. Assim obtemos a solução $X_1=\begin{pmatrix}
		1\\1\\3\\1
		\end{pmatrix}$.
		
		Vamos checar se é solução:
		\[
		\begin{pmatrix}
		3&0&-1&0\\4&1&-1&-2\\0&1&0&-1
		\end{pmatrix}\begin{pmatrix}
		1\\1\\3\\1
		\end{pmatrix}=\begin{pmatrix}
		3-3 \\ 4+1-3-2 \\ 1-1
		\end{pmatrix}=\begin{pmatrix}
		0\\0\\0
		\end{pmatrix},\] ou seja, $X_1$ é, de fato, solução.
	\end{enumerate}
\end{sol}\pagebreak

\begin{qst}
	Considere a matriz $A=\begin{pmatrix}
	1&1&1\\2&5&-2\\1&7&-7
	\end{pmatrix}$ e faça o que se pede:
	\begin{enumerate}[a)]
		\item Resolva o sistema linear homogêneo $AX=0$.
		\item Dada a matriz coluna $X_1=\begin{pmatrix}
		1\\1\\1
		\end{pmatrix}$ encontre uma matriz $B$ tal que $X_1$ seja solução do sistema $AX=B$.
		\item Encontre, sem fazer contas, outra solução qualquer do sistema $AX=B$ e mostre que ela é solução.
	\end{enumerate}
\end{qst}
\begin{sol}
	\begin{enumerate}[a)]
		\item Para resolver o sistema linear homogêneo, vamos escalonar $A$:
		\[\begin{array}{rl}
		\begin{pmatrix}
		1&1&1\\2&5&-2\\1&7&-7
		\end{pmatrix}&\rightsquigarrow\begin{pmatrix}
		1&1&1\\0&3&-4\\0&6&-8
		\end{pmatrix}\cr\cr
		\rightsquigarrow\begin{pmatrix}
		1&1&1\\0&1&-4/3\\0&6&-8
		\end{pmatrix}&\rightsquigarrow\begin{pmatrix}
		1&0&7/3\\0&1&-4/3\\0&0&0
		\end{pmatrix}
		\end{array}\]agora vemos que temos equações $x=-7z/3$ e $y=4z/3$. Como $x$ e $y$ dependem de $z$, vamos declarar $z$ como variável livre e escrever
		\[S=\left\{(x,y,z)\in\R^3\mid x=-7z/3\mbox{ e }y=4z/3\right\}\] (ou
		\[S=\left\{\lambda\begin{pmatrix}
		-7/3\\4/3\\1
		\end{pmatrix}\mid \lambda\in \R\right\}\]se você preferir).
		
		\item Como a matriz $X_1$ é solução do sistema $AX=B$, por definição de solução sabemos que $AX_1=B$. Mas já temos $A$ e $X_1$, assim podemos calcular $B$:
		\[B=AX_1=\begin{pmatrix}
		1&1&1\\2&5&-2\\1&7&-7
		\end{pmatrix}\begin{pmatrix}
		1\\1\\1
		\end{pmatrix}=\begin{pmatrix}
		1+1+1\\2+5-2\\1+7-7
		\end{pmatrix}=\begin{pmatrix}
		3\\5\\1
		\end{pmatrix}.\] Ou seja, a única matriz $B$ tal que $X_1$ é solução do sistema $AX=B$ é a matriz $\begin{pmatrix}
		3\\5\\1
		\end{pmatrix}$.
		\item Já sabemos que a soma de uma solução não-trivial do sistema homogêneo $AX=0$ com uma solução do sistema $AX=B$ também é solução do sistema $AX=B$. Então, escolhendo $\lambda=1$ temos que $X_0=\begin{pmatrix}
		-7/3\\4/3\\1
		\end{pmatrix}$ é solução não-trivial do sistema homogêneo, donde $X_0+X_1=\begin{pmatrix}
		-7/3\\4/3\\1
		\end{pmatrix}+\begin{pmatrix}
		1\\1\\1
		\end{pmatrix}=\begin{pmatrix}
		-4/3\\7/3\\2
		\end{pmatrix}$ é solução do sistema $AX=B$. Vamos verificar:
		\[\begin{pmatrix}
		1&1&1\\2&5&-2\\1&7&-7
		\end{pmatrix}\begin{pmatrix}
		-4/3\\7/3\\2
		\end{pmatrix}=\begin{pmatrix}
		-4/3+7/3+2\\-8/3+35/3-4\\-4/3+49/3-14
		\end{pmatrix}=\begin{pmatrix}
		3\\5\\1
		\end{pmatrix}=B,\]como queríamos mostrar.
	\end{enumerate}
\end{sol}\pagebreak

\begin{qst}
	Dadas as matrizes $A=\begin{pmatrix}
	1&2&3\\1&1&2\\0&1&2
	\end{pmatrix}$ e $B=\begin{pmatrix}
	5\\7\\-2
	\end{pmatrix}$, faça o que se pede:
	\begin{enumerate}[a)]
		\item Calcule $A^{-1}$.
		\item Calcule $A^{-1}B$.
		\item Exiba o conjunto solução do sistema linear $AX=B$.
	\end{enumerate}
\end{qst}
\begin{sol}
	\begin{enumerate}[a)]
		\item Para achar $A^{-1}$ temos que resolver o sistema $AX=I$. Vamos lá:
		\[\begin{array}{rl}
		\begin{augmatrix}{ccc:ccc}
		1&2&3&1&0&0\\
		1&1&2&0&1&0\\
		0&1&2&0&0&1
		\end{augmatrix}&\rightsquigarrow\begin{augmatrix}{ccc:ccc}
		1&2&3&1&0&0\\
		0&-1&-1&-1&1&0\\
		0&1&2&0&0&1
		\end{augmatrix}\cr\cr
		\rightsquigarrow\begin{augmatrix}{ccc:ccc}
		1&2&3&1&0&0\\
		0&1&1&1&-1&0\\
		0&1&2&0&0&1
		\end{augmatrix}&\rightsquigarrow\begin{augmatrix}{ccc:ccc}
		1&0&1&-1&2&0\\
		0&1&1&1&-1&0\\
		0&0&1&-1&1&1
		\end{augmatrix}\cr\cr
		\rightsquigarrow\begin{augmatrix}{ccc:ccc}
		1&0&0&0&1&-1\\
		0&1&0&2&-2&-1\\
		0&0&1&-1&1&1
		\end{augmatrix}
		\end{array}\]e vemos que $A^{-1}=\begin{pmatrix}
		0&1&-1\\2&-2&-1\\-1&1&1
		\end{pmatrix}$. Vamos conferir:
		\[\begin{pmatrix}
		1&2&3\\1&1&2\\0&1&2
		\end{pmatrix}\begin{pmatrix}
		0&1&-1\\2&-2&-1\\-1&1&1
		\end{pmatrix}=\begin{pmatrix}
		4-3 & 1-4+3 & -1-2+3\\
		2-2 & 1-2+2 & -1-1+2\\
		2-2 & -2+2 & -1+2
		\end{pmatrix}=I\]
		\[\begin{pmatrix}
		0&1&-1\\2&-2&-1\\-1&1&1
		\end{pmatrix}\begin{pmatrix}
		1&2&3\\1&1&2\\0&1&2
		\end{pmatrix}=\begin{pmatrix}
		1 & 1-1 & 2-2\\
		2-2 & 4-2-1 & 6-4-2\\
		-1+1 & -2+1+1 & -3+2+2
		\end{pmatrix}=I\]ou seja, elas são mesmo inversas.
		\item Vamos calcular $A^{-1}B$:
		\[A^{-1}B=\begin{pmatrix}
		0&1&-1\\2&-2&-1\\-1&1&1
		\end{pmatrix}\begin{pmatrix}
		5\\7\\-2
		\end{pmatrix}=\begin{pmatrix}
		7+2\\10-14+2\\-5+7-2
		\end{pmatrix}=\begin{pmatrix}
		9\\-2\\0
		\end{pmatrix}\]logo $A^{-1}B=\begin{pmatrix}
		9\\-2\\0
		\end{pmatrix}$.
		\item Como $A$ é inversível, o sistema $AX=B$ possui solução única dada por $A^{-1}B$, ou seja,
		\[S=\left\{\begin{pmatrix}
		9\\-2\\0
		\end{pmatrix}\right\}.\]
	\end{enumerate}
\end{sol}
\pagebreak

\begin{qst}
	Dadas as matrizes $A=\begin{pmatrix}
	1&1\\0&1
	\end{pmatrix}$ e $B=\begin{pmatrix}
	0&1\\1&0
	\end{pmatrix}$, faça o que se pede:
	
	\begin{enumerate}[a)]
		\item Calcule $AB$ e $BA$ e compare os resultados obtidos.
		\item Escolha $A$ ou $B$. Você seria capaz de encontrar uma matriz $C$ que comute com a matriz que você escolheu (por exemplo, se você escolher a matriz $A$, uma matriz $C$ tal que $AC=CA$)? Explique seu raciocínio.
	\end{enumerate}
\end{qst}
\begin{sol}
	\begin{enumerate}[a)]
		\item Vamos calcular:
		\[AB=\begin{pmatrix}
		1&1\\0&1
		\end{pmatrix}\begin{pmatrix}
		0&1\\1&0
		\end{pmatrix}=\begin{pmatrix}
		1 & 1\\1&0
		\end{pmatrix}\]
		\[BA=\begin{pmatrix}
		0&1\\1&0
		\end{pmatrix}\begin{pmatrix}
		1&1\\0&1
		\end{pmatrix}=\begin{pmatrix}
		0&1\\1&1
		\end{pmatrix}.\]
		
		Comparando os resultados obtidos, vemos que $AB\neq BA$.
		\item A resposta é sim, independente de ter escolhido $A$ ou $B$. Por exemplo, você poderia escolher $C=0$ ou $C=I$ e, nesse caso, certamente $AC=CA$ e $BC=CB$. Além disso, poderia notar que $A$ e $B$ são inversíveis e, portanto, $AA^{-1}=A^{-1}A=I$ e $BB^{-1}=B^{-1}B=I$, ou seja, poderíamos tomar $C$ como sendo a inversa da matriz escolhida. Ou você poderia encontrar alguma matriz $C$ aleatória da sua cabeça que comute com $A$ ou $B$.
		
		O importante aqui é que você saiba que \textbf{o produto em geral não é comutativo, mas isso não significa que ele nunca é comutativo}.
	\end{enumerate}
\end{sol}\pagebreak

\begin{bonus}
	Considere a reação
	\[x{\rm Fe_3O_4}+y{\rm CO}\rightarrow z{\rm FeO}+w{\rm CO_2},\]em que $x,y,z,w\in \R$ são coeficientes de balanceamento, e faça o que se pede:
	\begin{enumerate}[a)]
		\item Escreva as equações que relacionam as quantidades de cada substância na reação - uma equação para o ferro, uma equação para o oxigênio e uma equação para o carbono (por exemplo, como temos $3x$ átomos de ferro do lado esquerdo e $z$ átomos de ferro do lado direito, isso nos dá a equação $3x=z$).
		\item Use as equações que obtidas no item anterior para montar um sistema linear homogêneo.
		\item Resolva o sistema obtido no item anterior.
		\item Escolha uma solução não-trivial do sistema e verifique que ela é um balanceamento da reação.
		\item Conclua descrevendo um procedimento sistemático para balanceamento de reações.
	\end{enumerate}
\end{bonus}
\begin{sol}
	\begin{enumerate}[a)]
		\item Comparando o ferro obtemos $3x=z$, comparando o oxigênio obtemos $4x+y=z+2w$ e comparando o carbono obtemos $y=w$.
		\item Podemos reescrever todas essas equações como $3x-z=0$, $4x+y-z-2w=0$ e $y-w=0$ e montar o sistema linear homogêneo
		\[\begin{pmatrix}
		3&0&1&0\\4&1&-1&-2\\0&1&0&-1
		\end{pmatrix}\begin{pmatrix}
		x\\y\\z\\w
		\end{pmatrix}=\begin{pmatrix}
		0\\0\\0
		\end{pmatrix}.\]
		\item Ao invés de resolver o sistema, note que é o mesmo sistema da Questão 1, e já sabemos a solução:\[S=\left\{\lambda \begin{pmatrix}
		1\\1\\3\\1
		\end{pmatrix}\mid \lambda\in \R \right\}.\]
		\item Podemos novamente escolher a solução $X_1=\begin{pmatrix}
		1\\1\\3\\1
		\end{pmatrix}$ e ver que a reação
		\[{\rm Fe_3O_4}+{\rm CO}\rightarrow 3{\rm FeO}+{\rm CO_2}\]está balanceada.
		\item Dada uma reação qualquer, podemos atribuir variáveis aos coeficientes de balanceamento e, comparando ambos os lados, obter equações relacionando essas variáveis.
		
		Com essas equações podemos montar um sistema linear homogêneo.
		
		Resolvendo esse sistema, obtemos um conjunto de soluções que podemos facilmente verificar que são balanceamentos da reação original.
	\end{enumerate}
\end{sol}
\end{document}